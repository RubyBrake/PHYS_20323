
%%%%%%%%%%%%%%%%%%%%%%%%%%%%%%%%%%%%%%%%%%%%%%%%%%%%%%%%%%%%
\documentclass[12pt]{article}
\usepackage{fancyhdr}
\usepackage{pslatex}
\usepackage{epsfig}
\usepackage{times}
\usepackage{amsmath}
\usepackage{mathrsfs}
\usepackage[dvipsnames]{xcolor}
\usepackage[hidelinks]{hyperref}%renewcommand{\topfraction}{1.0}
\renewcommand{\topfraction}{1.0}
\renewcommand{\bottomfraction}{1.0}
\renewcommand{\textfraction}{0.0}
\setlength {\textwidth}{6.6in}
\hoffset=-1.0in
\oddsidemargin=1.00in
\marginparsep=0.0in
\marginparwidth=0.0in                                                                               
\setlength {\textheight}{9.0in}
\voffset=-1.00in
\topmargin=1.0in
\headheight=0.0in
\headsep=0.00in
\footskip=0.50in                                         
\setcounter{page}{3}
\begin{document}
\def\pos{\medskip\quad}
\def\subpos{\smallskip \qquad}
\newfont{\nice}{cmr12 scaled 1250}
\newfont{\name}{cmr12 scaled 1080}
\newfont{\swell}{cmbx12 scaled 800}
%%%%%%%%%%%%%%%%%%%%%%%%%%%%%%%%%%%%%%%%%%%%%%%%%%%%%%%%%%%%
%     DO NOT CHANGE ANYTHING ABOVE THIS LINE
%%%%%%%%%%%%%%%%%%%%%%%%%%%%%%%%%%%%%%%%%%%%%%%%%%%%%%%%%%%%
%     DO NOT CHANGE ANYTHING ABOVE THIS LINE
%%%%%%%%%%%%%%%%%%%%%%%%%%%%%%%%%%%%%%%%%%%%%%%%%%%%%%%%%%%%
%     DO NOT CHANGE ANYTHING ABOVE THIS LINE
%%%%%%%%%%%%%%%%%%%%%%%%%%%%%%%%%%%%%%%%%%%%%%%%%%%%%%%%%%%%

% Title
\begin{center}
{\large
PHYS  20323/60323: Fall 2025 - LaTeX Example
}\\
\end{center}

% line 1
\noindent {1. At time t = 0 a particle is represented by the wave function} 
% placeholder for the fancy bracket stuff
\begin{equation}
\Psi (x) = \begin{cases}
\vspace{12pt}
A \frac{x}{a}, &  0 < x < a \\
\vspace{12pt}
A \frac{(b-x)}{(b-a)},& a < x < b \\
0, & \text{otherwise}
\end{cases}
\end{equation} 

% where A, a, and b are constants
where \textit{A, a,} and \textit{b} are constants. \\

(a)  (3.3 points)  Normalize $\Psi$ (i.e., find \textit{A} in terms of \textit{a} and \textit{b}).\\

(b) (3.3 points) Where is the particle likely to be found at \textit{t} = 0? \\

(c) (3.4 points) What is the expectation value of \textit{x}? \\

% number 2
\noindent {2. \bf The following questions refer to stars in the Table below.} \\
\textit{Note: there may be multiple answers.}

%Come back and fix stuff later!!!!!!!!!!!!!!!!!!!!!!
\begin{center}

\begin{tabular} {|c|c|c|c|c|c|c|}
\hline Name & Mass & Luminosity & Lifetime & Temperature & Radius & Variable? \\
\hline $\delta$ Scu. & 2.0 $\textit{M}_{\bigodot}$ &  & 5.0 $ \times 10^8$ years &  & 2.0 $\textit{R}_{\bigodot}$ & Y \\
\hline $\gamma$ Del. & 0.7 $\textit{M}_{\bigodot}$ &  & 4.5 $\times 10^{10}$ years & 5000 K &  & N \\
\hline $\beta$ Cyg. & 1.3 $\textit{M}_{\bigodot}$ & 3.5 $\textit{L}_{\bigodot}$ &  &  &  & Y \\
\hline $\eta$ Car. & 60. $\textit{M}_{\bigodot}$ & $10^6$ $\textit{L}_{\bigodot}$ & 8 $\times 10^5$ years & & & Y \\
\hline $\epsilon$ Eri. & 6.0 $\textit{M}_{\bigodot}$ & $10^3$ $\textit{L}_{\bigodot}$ & & 20,000 K & & N \\
\hline $\alpha$ Cen. & 1.0 $\textit{M}_{\bigodot}$ & & & 6000 K & 1.0 $\textit{R}_{\bigodot}$ & N \\
\hline

\end{tabular}

\end{center}

(a) (4 points) Which of these stars will produce a planetary nebula.\\

(b) (4 points) Elements heavier than \texttt{Carbon} will be produced in which stars.\\

% numba threeeee
\noindent{3. An electron is found to be in the spin state (in the \textit{z}-basis): $\chi = A(\substack{3i\\4})$}\\

(a) (5 points) Determine the possible values of \textit{A} such that the state is normalized.\\

(b) (5 points) Find the expectation values of the operators {\color{red}$S_x$}, {\color{purple}$S_y$}, {\color{orange}$S_z$} and $\textit{S}^{\xrightarrow{}2}$\\

The matrix representations in the \textit{z}-basis for the components of electron spin operators are \\ $\phantom{asl}$ given by: \\

%the colorful stuff at the bottom

{\color{red} $\mathbf{S_x} = \frac{h}{2}(\substack{0\\1} \hspace{12pt} \substack{1\\0})$ ;} \hspace{35pt} {\color{purple} $\mathbf{S_y} = \frac{h}{2}(\substack{0\\i} \hspace{12pt} \substack{-i\\0})$ ;} \hspace{35pt}  {\color{orange}$\mathbf{S_z} = \frac{h}{2} (\substack{1\\0} \hspace{12pt} \substack {0\\-1})$}


% This goes at the end I think
\end{document}